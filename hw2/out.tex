\documentclass[12pt,a4paper]{article}

\title{作業二:加密}
\author{學號:C24106082\\
姓名:陳宏彰}
\date{2022-09-20}

\usepackage{amsmath} 
\usepackage{mathtools}
\usepackage{graphicx}
\usepackage{array}   % for \newcolumntype macro
\newcolumntype{L}{>{$}l<{$}} % math-mode version of "l" column type
\newcolumntype{C}{>{$}c<{$}} % math-mode version of "l" column type
\newcolumntype{R}{>{$}r<{$}} % math-mode version of "l" column type
\usepackage[utf8]{inputenc}
\usepackage[T1]{fontenc}
\usepackage{textcomp}
\usepackage{gensymb}
\usepackage{multirow}
\usepackage{hhline}
\usepackage{indentfirst}
\usepackage{hyperref}
\hypersetup{
    colorlinks,
    linkcolor={red!50!black},
    citecolor={blue!50!black},
    urlcolor={blue!80!black}
}
\usepackage{footnote}
\makesavenoteenv{tabular}
\usepackage{listings}

\usepackage[CJKmath=true,AutoFakeBold=3,AutoFakeSlant=.2]{xeCJK} 
\newCJKfontfamily\Kai{[jf-openhuninn-1.1.ttf]}       
\newCJKfontfamily\Hei{[jf-openhuninn-1.1.ttf]}   
\newCJKfontfamily\NewMing{[jf-openhuninn-1.1.ttf]} 

\usepackage{fontspec}
\setmainfont{[jf-openhuninn-1.1.ttf]}
\setmonofont{[jf-openhuninn-1.1.ttf]}
\XeTeXlinebreaklocale "zh"

\newcommand{\overbar}[1]{\mkern 1.5mu\overline{\mkern-1.5mu#1\mkern-1.5mu}\mkern 1.5mu}

\begin{document}
\maketitle  

\section{原理}
\subsection{加密}
金鑰的六個數字依序為 $a$ $b$ $c$ $d$ $e$ $f$,明文的每兩個字為 $x$ $y$,帶入 $X=ax+by+c$ $Y=dx+ey+f$,得到的數字再用 62 進位編碼($0-9a-ZA-Z$)。在輸出的時候再加上每個字加密後的位數用 62 位元編碼當前綴,例如 $4Y$ $8R$ 會各加上 $2$ 後再串起來,變成 $24Y28R$,全部串起來就得到密文
    
\subsection{解密}
依照位數前綴將密文差開、轉成 10 進位用克拉瑪公式解出 $x$ $y$,就得到原本的明文

\section{使用範例}
\subsection{加密}
\begin{lstlisting}
$ go run . encrypt 1 2 3 4 5 6 'ncku information security'
2512fb25y2gr23X2aD2572fq25w2gs24W2eX25j2g325o2g824h2br24S2
eB25C2gQ25u2ge2322au
\end{lstlisting}

\subsection{解密}
\begin{lstlisting}
$ go run . decrypt 1 2 3 4 5 6 '2512fb25y2gr23X2aD2572fq25
w2gs24W2eX25j2g325o2g824h2br24S2eB25C2gQ25u2ge2322au'
ncku information security
\end{lstlisting}

\section{程式碼}
完整程式碼在 \url{https://github.com/simbafs/NCKU-IS-HW2},以下僅擷取部份

\subsection{加密}
\begin{lstlisting}
func Encrypt(e1, e2 *Equation, secret string) string {
	if len(secret)%2 == 1 {
		secret += " "
	}

	var encrypted string
	for i := 0; i < len(secret); i += 2 {
		x, y := secret[i], secret[i+1]
		X, Y := e1.CalcText(x, y), e2.CalcText(x, y)
		// fmt.Println(x, y, X, Y)
		lX, lY := to62(int64(len(X))), to62(int64(len(Y)))
		encrypted += lX + X + lY + Y
	}

	return encrypted
}
\end{lstlisting}

\subsection{解密}
\begin{lstlisting}
func Decrypt(e1, e2 *Equation, secret string) string {
	s := ""

	for i := 0; i < len(secret); {
		lX := toDec(string(secret[i]))
		i++
		X := toDec(secret[i : i+int(lX)])
		// fmt.Println("\t")
		i += int(lX)
		lY := toDec(string(secret[i]))
		i++
		Y := toDec(secret[i : i+int(lY)])
		i += int(lY)
		x, y := e1.Solve(e2, X, Y)
		// fmt.Println(lX, X, lY, Y, x, y)
		s += string(x) + string(y)
	}

	return s
}
\end{lstlisting}

\end{document}
